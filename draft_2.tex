
\documentclass[11pt]{article}
\def\baselinestretch{1}
\usepackage{amssymb}
\usepackage{amsmath}
\usepackage{amsfonts}
\usepackage{graphicx}
\usepackage{color}
\usepackage{amsmath,amsthm,amsfonts,epsfig,setspace}
\numberwithin{equation}{section}

\usepackage{multicol}
\usepackage{stmaryrd}
\usepackage{pgf}
\usepackage{pgffor}
\usepgflibrary{plothandlers}
\usepackage{pgfplots}
\pgfplotsset{compat=newest}
 \pgfplotsset{width=15cm}
\pgfplotsset{plot coordinates/math parser=false}
\newlength\figureheight
\newlength\figurewidth



\newcommand{\ds}{\displaystyle}
\def\nm{\noalign{\medskip}}
\newtheorem{theorem}{Theorem}[section]
\newtheorem{remark}{Remark}[section]
\newtheorem{corollary}{Corollary}[section]
\newtheorem{definition}{Definition}[section]
\newtheorem{lemma}{Lemma}[section]
\newtheorem{assumption}{Assumption}[section]
\newtheorem{proposition}{Proposition}[section]
\newtheorem{conjecture}{Conjecture}[section]
\newtheorem{example}{Example}
\newtheorem{cond}{Condition}
\setlength\topmargin{-1cm} \setlength\textheight{220mm}
\setlength\oddsidemargin{0mm}
\setlength\evensidemargin\oddsidemargin \setlength\textwidth{160mm}
\setlength\baselineskip{18pt}

%\usepackage{showkeys}

\def\no{\noindent} \def\dr{\partial}
\def \Vh0{\stackrel{\circ}{V}_h} \def\to{\rightarrow}
\def\cl{\centerline}   \def\ul{\underline}


%\def\D{\%end{document}}

\def\b{\beta}  \def\a{\alpha} \def\eps{\varepsilon}
\def\G{\Gamma}
\def\K{\texttt{K}}
\def\e{\eta}
\def\m{\mbox} \def\t{\times}  \def\lam{\lambda}
\def\ms{\medskip} \def\bs{\bigskip} \def\ss{\smallskip}
%\def\Box{\sharp}
\def\Box{\square}
\def\l|{\left|}
\def\r|{\right|}

\def\restriction#1#2{\mathchoice
              {\setbox1\hbox{${\displaystyle #1}_{\scriptstyle #2}$}
              \restrictionaux{#1}{#2}}
              {\setbox1\hbox{${\textstyle #1}_{\scriptstyle #2}$}
              \restrictionaux{#1}{#2}}
              {\setbox1\hbox{${\scriptstyle #1}_{\scriptscriptstyle #2}$}
              \restrictionaux{#1}{#2}}
              {\setbox1\hbox{${\scriptscriptstyle #1}_{\scriptscriptstyle #2}$}
              \restrictionaux{#1}{#2}}}
\def\restrictionaux#1#2{{ #1 \smash{\vrule height 1.1\ht1 depth .95\dp1}}_{\,#2}}




\newcommand{\cqfd}{\hfill $\square$\\ \medskip}
\newcommand{\R}{\mathbb{R}}
\newcommand{\N}{\mathbb{N}}
\newcommand{\Z}{\mathbb{Z}}
\newcommand{\C}{\mathbb{C}}

\newcommand{\nubf}{\boldsymbol{\nu}}
\newcommand{\phibf}{\boldsymbol{\varphi}}

\newcommand{\fbf}{\mathbf{f}}
\newcommand{\wbf}{\mathbf{w}}
\newcommand{\xbf}{\mathbf{x}}
\newcommand{\vbf}{\mathbf{v}}
\newcommand{\ebf}{\mathbf{e}}
\newcommand{\gbf}{\mathbf{g}}

\newcommand{\dbf}{\mathbf{d}}
\newcommand{\ybf}{\mathbf{y}}
\newcommand{\kbf}{\mathbf{k}}
\newcommand{\Lbf}{\mathbf{L}}
\newcommand{\Ubf}{\mathbf{U}}
\newcommand{\Ibf}{\mathbf{I}}
\newcommand{\Wbf}{\mathbf{W}}
\newcommand{\Xbf}{\mathbf{X}}
\newcommand{\Ebf}{\mathbf{E}}



\newcommand{\dd}{\mathrm{d}}
\newcommand{\tin}{\text{ in }}
\newcommand{\ton}{\text{ on }}
\newcommand{\tif}{\text{ if }}


\definecolor{tagada}{RGB}{235,23,55} 
\definecolor{vertperso}{RGB}{0,125,0} 

\title{Modal analysis of resonant dielectric nano-structures}




\begin{document}
\maketitle


\section{Introduction}

\begin{equation}
\label{eq:transverse} 
 \left\{
\begin{array} {ll}
&\ds\Delta u+ \omega^2 (1+n(x)) u  = 0 \quad \mbox{in } \R^2 \backslash \partial D, \\
\nm
&  u^s:= u- u^{i}  \,\,\,  \mbox{satisfies the Sommerfeld radiation condition}.
  \end{array}
 \right.
\end{equation}
with
\begin{align}
n(x)= \eta\chi(\bar{D}).
\end{align}


\subsection{Integral formulation}

Denote by $\Gamma^\omega$ the Green's function associated with Helmholtz' equation in the free space in the medium:
\begin{align*}
\left( \Delta +\omega^2\right) \Gamma(x,y)= \delta_y(x),\qquad (x,y)\in \R^d\times \R^d
\end{align*}

\begin{proposition}{Lippman-Schwinger equation}

\begin{align*}
u-u^i(x) = -\omega^2\eta \int_D u(y) \Gamma^\omega(x,y) \dd y
\end{align*}
\end{proposition}

\begin{definition}
\begin{align*}
\mathcal{K} \begin{aligned} L^2(D)& \longrightarrow L^2(D) \\
f &\longmapsto - \int_D f(y) \Gamma^\omega (\cdot,y) \cdot  \dd y
\end{aligned}
\end{align*}
\end{definition}

\begin{proposition}
\begin{align}\label{eq:lippman}
\left( I-\omega^2 \eta \mathcal{K}\right)\left[ u\big\vert_D\right] = u^i \qquad \tin D
\end{align}
\end{proposition}
\subsection{Spectral analysis of $\mathcal{K}_D$}

The following lemmas are from \cite{ammari2015super}. 


\begin{lemma}
The operator $\mathcal{K}$ is compact from $L^2(D)$ to $L^2(D)$. In fact, $\mathcal{K}$ is bounded from $L^2(D)$ to $H^2(D)$. Moreover, $\mathcal{K}$ is a Hilbert-Schmidt operator.
\end{lemma}

\begin{lemma}
Let $\sigma(\mathcal{K})$ be the spectrum of $\mathcal{K}$:
\begin{enumerate}
\item $\sigma(\mathcal{K}) =\{0,\lambda_1,\ldots,\lambda_n,\ldots\}$ where $\vert \lambda_1\vert \geq \vert \lambda_2\vert \geq \cdots$ and $\lambda_n\rightarrow 0$.
\item $\sigma(\mathcal{K})\setminus\sigma_p(\mathcal{K})=\{0\}$, $\sigma_p$ being the point spectrum of $\mathcal{K}$.
\end{enumerate}
\end{lemma}

\begin{lemma}Let $H_j$ denote the generalized eigenspaces of the operator $\mathcal{K}$. Then the following decomposition holds
\begin{align*}
L^2(D)= \overline{\mathop{\bigcup}_{j=1}^\infty H_j}
\end{align*}
\end{lemma}

\begin{lemma}There exists a basis $\{u_j,l,k\}$, $1\leq l \leq m_j$, $1\leq k \leq n_{j,l}$ for $H_j$ such that 
\begin{align*}
\mathcal{K}[u_{j,l,k}]= \lambda_j u_{j,l,k }+u_{j,l,k-1} \quad \text{for \ all\ } j,l,k \quad (u_{j,l,k}=0 \text{\ for \ } k\leq 0).
\end{align*}
$m_j$ is the geometric multiplicity of $\lambda_j$, given by the dimension of $N(\lambda_j I - \mathcal{K})$ and $\sum_l n_{j,l}= N_j$ is the algebraic multiplicity of $\lambda_j$.
In $H_j$, in the basis $\{u_j,l,k\}$, $\mathcal{K}$ has the Jordan block representation:
\begin{align*}
K_D= \begin{pmatrix}
J_{j,1} & \ &\ \\
 & \ddots & \\
 &  & J_{j,m_j} 
\end{pmatrix} \qquad \text{with } \quad J_{j,l}= \begin{pmatrix}
\lambda_j & 1 & & \\
 & \ddots & \ddots & \\
 &         &  \lambda_j & 1 \\
 &          &           & \lambda_j
\end{pmatrix}
\end{align*}

 \end{lemma}
 
\subsection{Non Normality, Exceptional points}
The operator $\mathcal{K}$ is not normal in $L^2(D)$.
This can have two consequences:
\begin{itemize}
\item the existence of generalized eigenspaces that are not eigenspaces (degenerate eigenvalues $\lambda_j$ such that  $n_{j,l}\geq 2$ for some $1\leq l \leq m_j$)
\item the lack of orthogonality between generalized eigenspaces
\end{itemize}

\begin{remark}
In the first situation we are in presence of an \emph{exceptional point}. This is a particular situation, that happens only for specific values of $\omega$. This situation is not stable under a small perturbation of the frequency. Therefore, at first we will assume that there are no exceptional points. This specific case will be studied later.
\end{remark}

\begin{conjecture}
The set of $\omega$ such that $\mathcal{K}$ has an exceptional point is discrete.
\end{conjecture}

\begin{conjecture}
Exceptional points are of order $2$ or $4$ ?
\end{conjecture}




\section{Modal decomposition in the non exceptional case}

In this section, we make the following assumption:
\begin{assumption}
$\omega$ is such that $K$ has no exceptional eigenvalues, i.e. $n_{j,l}=1$ for every $j\in \N, 1\leq l \leq m_j$.
\end{assumption}

Denote $\Gamma := \{(j,l,k)\in \mathbb{N}^3,1\leq l \leq m_j$, $1\leq k \leq n_{j,l} \}$ the set of indices for the basis functions. Equip $\Gamma$ with the lexicographical order $\preceq$.
\begin{lemma}There exists an orthonormal basis $\{e_\gamma : \gamma \in \Gamma\}$ for $L^2(D)$ such that 
\begin{align*}
e_\gamma=\sum_{\gamma'\preceq \gamma} a_{\gamma,\gamma'} u_{\gamma'}
\end{align*}
\end{lemma}

\subsection{Modal expansion, resonance condition}

\begin{proposition}\label{prop:expansion}
\begin{align*}
u\big\vert_D (x) = \frac{1}{\tau \omega^2 \varepsilon_c} \sum_{\gamma \in \Gamma} \frac{1}{\frac{1}{\tau \varepsilon_c }-\omega^2 \lambda_j(\omega)}e_\gamma^i u_\gamma(x)
\end{align*}
with $e_\gamma^i = \int_D u^i(x) e_\gamma(x) \dd x$
\end{proposition}

\section{Temporal modal analysis}

Time dependent model, transverse electric case:
\begin{align*}
\varepsilon_D(x) \partial_t^2 u(x,t)-\Delta u(x,t) = 0
\end{align*}
with
\begin{align*}
u(x,t)=& u^s(x,t) +u^i(x,t) \\
u^i(x,t) =& \frac{1}{2\pi }\int_\R e^{-i\omega t} u^i(x,\omega) \dd \omega
\end{align*}

Using proposition  \ref{prop:expansion} we have:
\begin{align*}
u(x,t)= \frac{1}{2\pi }\int_\R e^{-i\omega t} \sum_\gamma e_\gamma^i (\omega) \left( I-\tau\varepsilon_c\omega^2 K_D(\omega)\right)[e_\gamma(\omega) ]\dd \omega
\end{align*}


\bibliographystyle{plain}
\bibliography{biblio}

\end{document}
